\chapter{Third-Party Code and Libraries}

%If you have made use of any third party code or software libraries, i.e. any code that you have not designed and written yourself, then you must include this appendix. 

%As has been said in lectures, it is acceptable and likely that you will make use of third-party code and software libraries. If third party code or libraries are used, your work will build on that to produce notable new work. The key requirement is that we understand what your original work is and what work is based on that of other people. 

%Therefore, you need to clearly state what you have used and where the original material can be found. Also, if you have made any changes to the original versions, you must explain what you have changed. 

%The following is an example of what you might say. 

%Apache POI library - The project has been used to read and write Microsoft Excel files (XLS) as part of the interaction with the client's existing system for processing data. Version 3.10-FINAL was used. The library is open source and it is available from the Apache Software Foundation 
%\cite{apache_poi}. The library is released using the Apache License 
%\cite{apache_license}. This library was used without modification. 

%Include as many declarations as appropriate for your work. The specific wording is less important than the fact that you are declaring the relevant work.

Boost C++ Libraries - This was used to parse and split string data from a YAML file. Version 1.69.0 was used. The library is open source and is available from the Boost Organisation.
\cite{boost}. The library is released under the Boost license
\cite{boost-license}. This library was used without modification.

yaml-cpp library - This was used to parse a configuration file using the YAML syntax. Version 0.6.2 was used. The library is open source and is available from the yaml-cpp Github Page.
\cite{yaml-cpp}. The library is released under the MIT license
\cite{yaml-cpp-license}. This library was used without modification.

ROS - This was used to interact with robotic systems also running ROS. The library is open source and is available from the ROS Website.
\cite{ros}. The library is released under the 3-Clause BSD License
\cite{bsd-license-3}. This library was used without modification.

ncurses - This was used to develop a text-based user interface. The library is open source and is available from the ncurses homepage.
\cite{ncurses}. The library is released under a permissive MIT-style license
\cite{ncurses-license}. This library was used without modification.
