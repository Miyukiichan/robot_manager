\chapter{Code Examples}
\label{appendix:c}

%For some projects, it might be relevant to include some code extracts in an appendix. You are not expected to put all of your code here - the correct place for all of your code is in the technical submission that is made in addition to the Project Report. However, if there are some notable aspects of the code that you discuss, including that in an appendix might be useful to make it easier for your readers to access. 

%As a general guide, if you are discussing short extracts of code then you are advised to include such code in the body of the report. If there is a longer extract that is relevant, then you might include it as shown in the following section. 

%Only include code in the appendix if that code is discussed and referred to in the body of the report. 

\section{Altering Child IO Signals}

\begin{verbatim}

/*These are a modified versioa of functions taken from the tty8 project
Copyright (c) 2015 Dima Krasner*/

/*Found in the Process class*/
int flags = fcntl(pty,F_GETFL);
fcntl(pty, F_SETSIG, SIGRTMIN + 1);
fcntl(pty, F_SETFL, O_ASYNC | O_NONBLOCK | flags);     
fcntl(pty, F_SETOWN, pid);

/*Found in main*/
void setMask(sigset_t * m, sigset_t * o) {
  sigemptyset(m);
  sigaddset(m, SIGCHLD);
  sigaddset(m, SIGINT);
  sigaddset(m, SIGTERM);
  sigaddset(m, SIGWINCH);
  sigaddset(m, SIGRTMIN);
  sigaddset(m, SIGRTMIN + 1);
}
\end{verbatim}

\section{Example Configuration File}

This YAML file was used as testing data throughout the project.

\begin{verbatim}
command:
  - name: "list"
    exec: "/bin/ls"
    param: "-l"
  - name: "example"
    exec: "example"
  - name: "echo"
    exec: "/bin/echo"
    param: "Hello World"
  - name: "scrolling"
    exec: "scrolltest"
    param: "50"
  - name: "turtle"
    exec: "/opt/ros/melodic/bin/rosrun"
    param: "turtlesim turtlesim_node"

teleop:
  topic: "turtle1/cmd_vel"
  degradation: 0.01
  turning_increment: 0.5
  linear_increment: 2
  refresh_timeout: 0.5
  degradation_timeout: 0.05

monitoring:
  telemetry_topic: "/idris/telemetry"
  safety_topic: "/idris/safety_status"
\end{verbatim}

\section{Test Programs}

The following are example programs used for testing the main application and were written separately from it.

\subsection{Example Program 01}

\begin{verbatim}
/*An example program used for stress testing the application*/

#include <stdio.h>
#include <unistd.h>

int main() {
  int i = 0;
  while(1) {
    printf("%d\n",i);
    i++;
  }
}
\end{verbatim}

\subsection{Example Program 02}

\begin{verbatim}
/*An example Program used to test scrolling functionality*/

#include <stdio.h>
#include <stdlib.h>

int main(int argc, char * argv[]) {
  if (argc < 2)
    printf("No args\n");
  else {
    for(int i = 0; i < atoi(argv[1]); i++)
      printf("%d\n",i);
  }
  return 0;
}
\end{verbatim}

\subsection{Example Program 03}

\begin{verbatim}
/*An example program used for testing file reading and YAML parsing*/

#include <yaml-cpp/yaml.h>
#include <iostream>
#include <string>
#include <vector>
#include "Process.h"
#include "FileReader.h"

main() {
  FileReader reader("test.yaml");
  std::vector<Process> processes = reader.getProcesses();
  processes[0].start();
  while(1) {
    processes[0].refreshBuffer(100);
    for (auto it : processes[0].getBuffer()) {
      std::cout << it;
    }
  }
}
\end{verbatim}
